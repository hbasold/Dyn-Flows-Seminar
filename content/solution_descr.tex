\section{Solution description}\label{sec:descr}
The next step is to find a description of the solution of \probRef{pathLift}
which in turn can be used to calculate it. Observe that want to continuously
deform a configuration and so need a construction to describe derivatives
of maps into $\Conf$. This leads to tangent spaces.

\subsection{Tangent spaces, derivatives and distributions}
The idea of the \term{tangent space} is essential for the rest of this discussion.
So we will build it up in two stages: first a geometric definition and then
the transfer to a more general setting.

\begin{definition}[Geometric tangent space]
    Let $M \subseteq \Rd$ be an n-dimensional manifold where $n \leq d$.
    The \term{geometric tangent space} at $p \in M$ is defined as
        \[ T_pM = \{p\} \times \R^n . \]
\end{definition}

\begin{remark}
    Elements of $\R^n$ should be viewed as vectors which have a direction
    length. A copy of $\R^n$ is then attached as a plane perpendicular to
    $p$ at $M$.
\end{remark}

We will not use the following definition but it is a starting point if one wants
work more on these topics. For a good introduction see \cite[Chapter 3]{JLee}
and \cite[Section 5B]{Kuehnel}.

\begin{definition}[Algebraic tangent space]
    Let $M$ be a smooth manifold. A linear map $X : \diff{1}{M}{\R} \rightarrow \R$
    is called a (directional) \term{derivation} or \term{tangent vector} at $p \in M$ if
        \[ X(fg) = f(p)Xg + g(p)Xf \]
    (chain rule). Set set of all those maps is $T_pM$ the \term{tangent space}
    at $p$. $TM$ denotes the disjoint union $\bigcup_{p \in M} \{p\} \times T_pM$.
\end{definition}

Here we define the tangent space for $\Conf$ as
    \[ T_z\Conf
        = \setDef{v \in \diffRdPw{0}{[0,L]}}
            {\inner{v(s)}{z(s)} = 0 \text{ for all } s \in [0,L]}.
    \]
It will be equipped with the induced inner product
    \[ \inner{v}{w} = \int_0^L \inner{v(s)}{w(s)} ds . \]

Obviously there are many more deformations of configuration possible than we need to
achieve our goal to let the snout follow a curve. We are only interested in those
deformations that require minimal ”energy” to let the snout go into a given direction.
So we only look at subsets of the tangent space $T_z\Conf$. This is where distributions
come into play (\cite[p. 8]{Rodriguez06}).

Informally a \term{distribution} or \term{generalized functions} defines
something like a derivative. Suppose we have a continuous function
$\varphi : \I \rightarrow M$ which is not everywhere differentiable,
so $\dot{\varphi}$ does not exist everywhere. Then we can relax this
to a generalized function which is not everywhere defined.
See \cite[p. 228ff]{EncyplopMath} and \cite[p. 257]{Treves}.
The definition of a distribution in a general setting is much more involved,
but is not needed for a geometric understanding. Good texts about distributions
are \cite[p. 158]{Boothby}, \cite[p. 34]{Chern} and \cite[p. 86ff]{Chevalley}.

Let us define a distribution for our problem:
\begin{definition}
    Let $\Delta_z \subseteq T_z\Conf$ be the set of all deformations $v$ of $z$
    where $\norm{v}$ is minimal when follwing a given direction $x \in T_{f(z)}\Rd$.
    The collection of all those $\Delta_z$ is denoted as $\Delta$.
\end{definition}

Since we are looking for a subset of all all those movements, we can refine the
path lifting from \refRemark{rem:path_lift}:
\begin{definition}[Horizontal]
    Let $\delta : \I \rightarrow \Conf$ be a $C^1$ curve. If
    $\dot{\delta}(t) \in \Delta_{\delta(t)}$ then $\delta$
    is called \term{horizontal}.
    
    Let $\gamma : \I \rightarrow \Rd$ be a $C^1$ curve.
    $\widetilde{\gamma} : [0,1] \rightarrow \Conf$ is called
    a \term{horizontal lift} if it is horizontal and
    $f \circ \widetilde{\gamma} = \gamma$.
\end{definition}

So we will study the following set:
\begin{definition}
    The \term{accessibility set} for a configuration $z \in \Conf$ is
    given by
    \[ \mathcal{A}(z) =
        \setDef{\gamma(1) \in \Conf}{\gamma : \I \rightarrow \Conf
            \text{ is horizontal and } \gamma(0) = z}.
    \]
\end{definition}

Before we go an we will formalize the distribution $\Delta$. For this we
need a map to go from the tangent space of the configurations to
the tangent space of the snout.

The following definition is again only given for better orientation in the literature.
\begin{definition}
    Let $M$ and $N$ be smooth manifolds and $F : M \rightarrow N$ a smooth map.
    For each $p \in M$ we define a map $T_pF : T_pM \rightarrow T_{F(p)}N$ by
        \[ (T_pFX)(g) = X(g \circ F) . \]
    It is called \term{tangent map} or \term{pushforward}.

    It satisfies the chain rule:
        \[ (T_pFX)(hg) = h(F(p))(T_pFX)(g) + g(F(p))(T_pFX)(h) . \]
\end{definition}

The geometric interpretation is as follows: if we follow $v$ in $T_zF(v)$, how
does the image of $F$ change. So the pushforward here is the Jacobian matrix,
the matrix of all partial derivatives. An exact geometric definition can
be found in \cite[Section 5B]{Kuehnel} and a good explanation in \cite[Chapter 3]{JLee}.

If we transfer this to $f : \Conf \rightarrow \Rd$ then
$T_zf : T_z\Conf \rightarrow T_{f(z)}\Rd = \Rd$ can be interpreted as: we take $z$
and deform it by $v \in T_z\Conf$, how does the snout move?
In a more computable manner:
    \[ T_zf(v) = \int_0^L v(s) ds .\]
    
So when does the snout \emph{not} move? For this recall the kernel of a function:
\begin{definition}
    Let $g : A \rightarrow B$ be a map between vector spaces. The \term{kernel}
    of $g$ is the set of all vector that are send to zero:
    \[ \ker g = \setDef{x \in A}{g(x) = 0} \]
\end{definition}

So $\ker T_zf$ is the set of all configuration changes where the snout does not move.
How does this help us to describe $\Delta$? Consider the orthogonal complement:
\begin{definition}
    Let $W$ be a space with an inner product. To vectors $x,y \in W$ are
    called \term{orthogonal} if $\inner{x}{y} = 0$.
    
    Let $U \subseteq W$. Then the \term{orthogonal complement} of $U$ is
    the set of vectors that are orthogonal to all vectors in $U$:
    \[U ^{\perp} = \setDef{y \in W}{\forall x \in U: \inner{x}{y} = 0} \]
    (see \cite[§62]{Halmos})
\end{definition}
% more infos: https://secure.wikimedia.org/wikipedia/en/wiki/Orthogonal_complement

\begin{definition}[Direct sum]
    Let $U$ and $V$ be vector spaces. The \term{direct sum}
    $W = U \oplus V$ is the Cartesian product $U \times V$
    with the induced operation
    \[ \alpha_1 (x_1,y_1) + \alpha_2 (x_2,y_2)
        = (\alpha_1 x_1 + \alpha_2 x_2, \alpha_1 y_1 + \alpha_2 y_2) \]
    (see \cite[§18]{Halmos})
\end{definition}

\begin{theorem}
    If $M$ is a subspace of $W$ of finite (co)dimension where $W$ has an
    inner product, then
    \[ W = M \oplus M^{\perp} \]
    
    See \cite[§66]{Halmos}, \cite[p.22]{Rodriguez07} and \cite[p.121]{Treves}.
\end{theorem}

% Example: R^2 = R_x \oplus R_y, x-/y-axis
% see: https://secure.wikimedia.org/wikipedia/en/wiki/Direct_sum#Internal_direct_sum

With this we can decompose $T_z\Conf$ into
    \[ T_z\Conf = \ker T_zf \oplus (\ker T_zf)^{\perp} .\]
Since $\ker T_zf$ means that the snout does not move, $\Delta_z = (\ker T_zf)^{\perp}$
achieves what we wanted.

% for later use:
%    If $f$ is differentiable for all $k \in \N$, then this class is denoted as
%    $\diffRd{\infty}{[0,L]}$ and f is called \term{smooth}.


\subsection{Möbius Transformations and the Möbius Group}
We are now going to discuss Möbius transformations which we will be using to to
describe the accessibility set. We will only discuss them in $\Rd$ but that should
be enough to build up some intuition to transfer it to $\Sd{d-1}$ where it will be
needed. The following is depicted from \cite[p. 20ff]{Beardon}.

Consider the sphere with center $a \in \Rd$ and radius $r > 0$ in $\Rd$
    \[ S(a,r) = \setDef{x \in \Rd}{\norm{x - a} = r} . \]
Let us define a \term{reflection} $\phi : \Rd \rightarrow \Rd$ at such a sphere.
In the case of the unit sphere $\Sd{d-1} = S(0,1)$ this is easily done by
    \[ \phi(x) = x/\norm{x}^2 . \]
That is norming $x$ to one and then pushing it out or pulling it in \resp by its
reciprocal length. Let us denote that operation by $x^* = x/\norm{x}^2$.
Using this notation the reflection can be easily extended to arbitrary spheres:
    \[ \phi(x) = a + r^2(x - a)^* . \]
Observe that this is not defined when $x = a$. This is overcome by extending
the space by an extra point which is not in $\Rd$
    \[ \RdEx = \Rd \cup \{\infty\} \]
and setting $\phi(a) = \infty$ since $\norm{\phi(x)} \rightarrow \infty$ when
$x \rightarrow a$. For the same reason we set $\phi(\infty) = a$.

Obviously $\phi : \RdEx \rightarrow \RdEx$ is now bijective and $\phi^2(x) = x$,
so it is its own inverse. Furthermore $\phi(x) = x$ iff $x \in S(a,r)$.

Another \term{reflection} can be done at a plane
    \[ P(a,t) = \setDef{x \in \Rd}{\inner{x}{a} = t} \cup \{\infty\} \]
by translating a point $x$ using
    \[ \phi(x) = x + \lambda a \]
where $\lambda$ is chosen \st $x + \frac{1}{2}\lambda a \in P(a,t)$.
The map is extended by $\phi(\infty) = \infty$ whereby $\infty$ lies in every plane.

Again $\phi$ is bijective and self-inverse and the only fix points are
again on the plane itself.

\begin{remark}
    Observe that $\RdEx$ is homeomorphic to $\Sd{d-1}$ using stereographic
    projection. So all this can be used directly on the unit sphere.
\end{remark}

\begin{definition}[Möbius Transformation]
    A \term{Möbius transformation} acting on $\RdEx$ is a finite composition
    of reflections on spheres and planes.
\end{definition}

The set of all Möbius transformations together with the function composition is a group.
This can be seen using the notes given above: every transformation is a
homeomorphism $\RdEx \rightarrow \RdEx$, the inverse of such a transformation
$\phi = \phi_1 \circ \cdots \circ \phi_n$ is
$\phi^{-1} = \phi_n \circ \cdots \circ \phi_1$.
Since $\phi^2(x) = x$ is the identity itself a Möbius transformation. This
group is called the \term{General Möbius group} $\GMRd$.

Let us write $\phi_1 \cdot \phi_2$ instead of $\phi_1 \circ \phi_2$ from now on.


A reflection $\phi$ is \term{orientation-reversing}. This roughly means that a triangle
$T$ with vertices $A,B,C$ labeled clockwise becomes a triangle $\phi(T)$ where the
vertices $\phi(A), \phi(B), \phi(C)$ are labeled anti-clockwise.

\begin{theorem}
    Reflections are orientation-reversing and angle-preserving.
    \begin{proof}
        See \cite[Theorem 3.1.6]{Beardon}.
    \end{proof}
\end{theorem}

So a composition of an even number of reflections is orientation-preserving. This
motivates

\begin{definition}[Möbius group]
    The \term{Möbius group} $\MRd$ is the subgroup of $\GMRd$ consisting of
    all orientation-preserving Möbius transformations.
\end{definition}

The important theorem is now
\begin{theorem}\label{theo:AccDescr}
    $\mathcal{A}(z) = \M(\Sd{d-1}) \cdot z$ for all $z \in \Conf$.
    \begin{proof}
        See \cite[Theorem 3.7]{Rodriguez07}.
    \end{proof}
\end{theorem}

\begin{remark}
    The notation $G \cdot g$ where $G$ is a group with its operation $\cdot$ and
    $g \in G$ denotes the subgroup
        \[ \setDef{a \cdot g \in G}{a \in G} \]
    of $G$. This can be extended to maps $z : X \rightarrow G$ by
        \[ (g \circ z)(s) = g \cdot z(s) . \]
    Which is called a \term{G-action} when
    \begin{align}
        & 1_G \circ z = z \\
        & (g \cdot h) \circ z = g \circ (h \circ z).
    \end{align}
    Which is obviously the case here. We reuse $\cdot$ here and write $g \cdot z$
    instead of $g \circ z$.
\end{remark}

There are now certain consequences of this theorem. The most important which is a first
step towards the solution of our problem is

% TODO: wirklich horizontal lift nennen oder doch lieber wie im Original?
\begin{corollary}\label{cor:liftDescr}
    Let $\widetilde{\gamma}(t) = z_t$ be a horizontal lift with
    $\widetilde{\gamma}(0) = z_0$. Then $z_t = g(t) \cdot z_0$
    where $g : [0,1] \rightarrow \M(\Sd{d-1})$ \st $g(0) = id$.
\end{corollary}

