\section{Nicht gleichförmig reduzierte zeiterweiterte Netzwerke}\label{sec:nonunif_cond}
Die gleichmäßige Unterteilung der Zeit durch $\timeDom = \{0,1, \ldots, T-1\}$ kann bei
anderen Problemen zu schlechten Ergebnissen führen. Wir wollen dazu kurz das Problem
des „Earliest Arrival Flows“ und dazu ein kritisches Beispiel betrachten.

TODO: Problem

TODO: Beispiel + Bild

Man kann dieses Problem umgehen, indem man den Zeitverlauf anders unterteilt. In dem
Beispiel würde die Unterteilung in Intervalle der Größe $\frac{1}{2}$
genügen ($\timeDom = \{0, \frac{1}{2}, 1\}$). Wir wollen uns für diesen Fall nur
die entsprechende Definition des zeitertweiterten Netzwerks ansehen.

\begin{definition}[Zeiterweitertes Netzwerk mit beliebigen Zeitintervallen]
    Sei $(\graph, \tau)$ und $L = (\theta_q)_{q \in R}$, wobei $R = \{0, \ldots, r\}$,
    sodass
    \[
        0 = \theta_0 < \theta_1 < \ldots < \theta_r < T.
    \]
    Setze $\theta_{r+1} = T$.
    Dann ist das \term{L-zeiterweiterte Netzwerk} $\tExp{L} = (V^L, \A^L)$
    gegeben durch
    \begin{itemize}
        \item $V^L = V \times R$ (schreibe $v_q = (v, q)$ und
            $V_q = \{v_q \in V^L\}$)
        \item $\A^L = \setDef{e_q = (v_q, u_{m(e, q)})}
                        {e = (v, u) \in \A, q \in R, \theta_q + \tau(e) \leq \theta_r}
                    \cup H$, \\
            \text{ wobei } $H = \setDef{(v_q, v_{q+1})}
                                    {v \in S_i \text{ und } q, q+1 \in R}$
    \end{itemize}

    Dabei ist $\func{m}{\A \times R}{R}$ definiert durch
    \[
        m(e, q) := \min \setDef{q' \in R}{\theta_q + \tau(e) \leq \theta_{q'}}.
    \]
    $m$ bestimmt also das Intervall, zwischen welchen die Kante $e$ transportiert.
    Der Rest ist analog wie bei den ursprünglichen zeiterweiterten Netzwerken
    definiert.
\end{definition}

\begin{remark}
    Offensichtlich ergibt sich mit einer Unterteilung $L = (0, 1, \ldots, T-1)$
    wieder $\tExp{T} = \tExp{L}$. Also sind L-zeiterweiterte Netzwerke
    eine Verallgemeinerung.
\end{remark}

TODO: wie werden diese benutzt?
