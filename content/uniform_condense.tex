\section{Gleichförmig reduzierte zeit-expandierte Netzwerke}\label{sec:unif_cond}

Zu QTP:
\begin{theorem}\label{theo:qtp_flow_ex}
    Sei $T \geq T^*$. Setze $\Delta := \frac{\eps^2 T}{n}$ und
    $T' := \lceil (1+\eps)^3 T / \Delta \rceil \Delta$.
    Dann gilt:
    \begin{enumerate}
        \item in $\redNetw{T'}{\Delta}$ existiert ein statischer Fluss $x$ mit
            Bedarf $|x| = (1 + \eps)D$ und Kosten $c(x) \leq (1+\eps)C$.
        \item aus $x$ kann ein dynamischer Fluss $f$ in $\graph$ berechnet werden,
            der bei $(1+\eps)T'$ endet und für den $|f| = D$ und $c(f) \leq C$
            gilt.
    \end{enumerate}
\end{theorem}

Bevor wir diesen Satz beweisen, folgern wir daraus den Algorithmus:

TODO:
\begin{theorem}
    QTP-FPTAS-Core
\end{theorem}

TODO: Einbettung in Suchframework

\begin{theorem}\label{theo:slow_flow}
    Sei $\func{f(e)}{\ropen{0, T + \delta}}{\R_+}$ ein dynamischer Fluss in $(\graph, \tau)$
    mit $|f| = D$ und $c(f) \leq C$ und einer endlichen Zerlegung in Flüsse $f(P)$
    auf Pfaden $P \in \pathSet$. Sei $\eps > 0$ und $(\graph, \tau')$ das
    gleiche Netzwerk mit geänderten Zeiten, für die
    $\left|\tau(e) - \tau'(e)\right| \leq \frac{\eps^2T}{n}$ gilt. Definiere
    einen Fluss in $(\graph, \tau')$
    \[
    \tilde{f}(P)(t) := \frac{1}{1+\eps} \frac{1}{\eps T}
                            \int_{t - \eps T}^{t} f(P)(\theta) \; d\theta \text{ .}
    \]
    Dann gilt für $\tilde{f}$:
    \begin{enumerate}
        \item $\tilde{f}$ endet bei $\delta + T + \eps T + \eps^2 T$
        \item $\tilde{f}$ verläuft nur auf Pfaden aus $\pathSet$
        \item $|\tilde{f}| = \frac{D}{1 + \eps}$
        \item $c(\tilde{f}) \leq \frac{C}{1 + \eps}$
    \end{enumerate}
    
    \begin{proof}
        TODO: in Tex und Anpassung an $T + \delta$.
    \end{proof}
\end{theorem}

\begin{lemma}\label{lem:relaxed_flow}
    Sei $\func{f^*(e)}{\ropen{0,T^*}}{\R_+}$ ein dynamischer Fluss in $(\graph, \tau)$
    mit Bedarf $|f^*| = D$ und Kosten $c(f^*) \leq C$. Für alle $\delta \geq 1$ und
    und $T \geq T^*$ existiert ein dynamischer Fluss
    $\func{f(e)}{\ropen{0, \delta T}}{\R_+}$ mit $|f| = \delta D$ und
    $c(f) \leq \delta C$. Dabei ist $f$ kreisfrei/kommt ohne Speicher aus, wenn $f^*$
    dies ist/tut.
    
    \begin{proof}
        TODO: in Tex
    \end{proof}
\end{lemma}

Zu QTP:
\begin{theorem}\label{theo:qtp_opt_flow}
    TODO: Existenz optimaler Fluss mit Pfadzerlegung und ohne Speicher
     (Corollary 4.5).
\end{theorem}

\begin{standaloneProof}[\theoRef{qtp_flow_ex}]
    \begin{enumerate}
        \item 
    Nach \theoRef{qtp_opt_flow} existiert ein optimaler Fluss $f^*$, der
    keinen Speicher benötigt und nur auf Pfaden verläuft.
    Setze $\delta = (1 + \eps)^2$ und wende \lemRef{relaxed_flow} mit
    $T \geq T^*$ an und erhalte einen Fluss $f$ der bei $(1 + \eps)^2 T$ endet.
    $f$ besitzt damit eine endliche Pfadzerlegung, hat Bedarf
    $|f| = (1 + \eps)^2 D$ und Kosten $c(f) \leq (1 + \eps)^2 C$.
    
    Setze nun $\tau'(e) := \lceil \tau(e) / \Delta \rceil \Delta$ für
    alle $e \in \A$. Dann ist offensichtlich
    $|\tau(e) - \tau'(e)| \leq \Delta = \frac{\eps^2 T}{n}$.
    Da $\delta T = T + 2 \eps T + \eps^2 T$ ist, liefert
    \theoRef{slow_flow} uns einen Fluss $\tilde{f}$ in $(\graph, \tau')$
    der bei
    \begin{align*}
        & T + 2 \eps T + \eps^2 T + \eps T + \eps^2 T \\
        & = T + 3 \eps T + 2 \eps^2 T \\
        & \leq T + 3 \eps T + 3 \eps^2 T + \eps^3 T \\
        & = (1 + \eps)^3 T \leq T'
    \end{align*}
    endet. $\tilde{f}$ hat Bedarf
    $|\tilde{f}| = \frac{|f|}{1+\eps} = (1 + \eps) D$ und ebenso
    Kosten $c(\tilde{f}) \leq (1 + \eps) C$.
    
    Da $\tau'$ und $T'$ nach Konstruktion ganzzahlig durch
    $\Delta$ teilbar sind, lässt sich $\tilde{f}$ als statischer Fluss $x$ in
    $\redNetw{T'}{\Delta}$ interpretieren, der die gewünschten
    Eigenschaften hat.
    
    \item ...
    
    \end{enumerate}
    
\end{standaloneProof}

