\section{Configuration space}\label{sec:conf}
%\begin{definition}[Topology]
%    Let $X$ be a set. If $\mathcal{O} \subseteq 2^X$ satisfies
%    \begin{enumerate}
%        \item $\forall \{U_{\alpha} | \alpha \in A\} \subseteq \mathcal{O}:
%                   \cup_\alpha U_\alpha \in \mathcal{O} $ 
%        \item $\forall U,V \in \mathcal{O}: U \cap V \in \mathcal{O}$
%        \item $\emptyset, X \in \mathcal{O}$
%    \end{enumerate}
%    then $\mathcal{O}$ is called a \term{topology} on $X$ and $X$ is called a
%    \term{topological space}. A set $U \in \mathcal{O}$ is called \term{open}.
%\end{definition}
%
%\begin{definition}[Continuous]
%    Let $X,Y$ be topological spaces and $f : X \rightarrow Y$ be a map.
%    If $f^{\leftarrow}(U) = \{x \in X | f(x) \in U \}$ is open in $X$, than
%    $f$ is called \term{continuous}.
%\end{definition}

%\begin{definition}[Piecewise differentiable]
%    Let $f : [0,L] \rightarrow \Rd, L \in \R_{>0}$ be a function and
%    $\mathcal{P} = \{0 = s_0 < s_1 < \ldots s_{N-1}, s_N = L\}$ a partition of $[0,L]$.
%    If all derivatives \[ \cfrac{d^j f_i}{d x^j} \]
%    exists and are continuous on for $j=0, \ldots, k$ and
%    $f(x) = (f_1(x), \ldots, f_d(x))$
%    then $f$ is called \term{k-times piecewise differentiable}. The class of all those
%    functions is denoted as $\diffRdPw{k}{[0,L]}$.
%\end{definition}

The class of all \term{k-times continuously differentiable} functions from
$[a,b] \subseteq \R$ to $\Rd$ is denoted as $\diffRd{k}{[a,b]}$. We refine this
class in the following sense:

\begin{definition}[Piecewise differentiable]
    Let $f : [0,L] \rightarrow \Rd, L \in \R_{>0}$ be a function. If there exists a
    partition $\mathcal{P} = \{0 = s_0 < s_1 < \ldots s_{N-1}, s_N = L\}$ of $[0,L]$
    \st $f_i = f|_{[s_i,s_{i+1}[} \in \diffRd{k}{[s_i,s_{i+1}[}$
    for all $i=0, \ldots, N-1$ and
    $f_i$ can be continuously extended to a map from $[s_i,s_{i+1}]$ to $\Rd$
    then $f$ is called \term{k-times piecewise continuously differentiable}.
    
    The class of all those functions is denoted by $\diffRdPw{k}{[0,L]}$.
\end{definition}

\begin{remark}
    As functions from $\diffRd{k}{[a,b]}$ have to be continuously differentiable
    $\diffRd{0}{[a,b]}$ denotes all continuous functions from $[a,b]$ to $\Rd$.
\end{remark}

To make clear what the length of snake means, we make the following definition:
\begin{definition}[Euclidean arc length]
    Let $\gamma \in \diffRdPw{1}{[0,L]}$. Then its (Euclidean) \term{arc length}
    is given by
    \[
        \arclength{\gamma}
        = \sum_{i=0}^{N-1}\int_{s_i}^{s_{i+1}} |\gamma'(t)| dt
    \]
    See \cite[p. 31]{Ratcliff}.
\end{definition}

With these definitions we can describe the snake charmer's problem more precisely:
\begin{definition}
    A \term{snake} is a map $S \in \diffRdPw{1}{[0,L]}, L \in \R_{>0}$
    with arc length $L$ which is rooted at the origin: $S(0) = 0$.
    
    So its derivatives exist everywhere except at $s_1, \ldots, s_{N-1}$ where
    they are extended by
    \[
        \left.\frac{dS}{ds}\right|_{s_i} = \lim_{s \downarrow s_i}\frac{dS}{ds}(s).
    \]
    
    $\lim_{s \downarrow s_i}$ denotes the limit of a function if s get arbitrary
    close to $s_i$ coming from values greater than $s_i$.

    $S(L)$ is called the \term{snout} of the snake.
\end{definition}

\begin{lemma}
    Observe that since $\arclength{S} = L$ it follows that
    \[
        \norm{\frac{dS}{ds}(\sigma)} = 1 \text{ for all } \sigma \in [0,L] .
    \]
    
    So it follows
    \[  \frac{dS}{ds}([0,L]) \subseteq \Sd{d-1} . \]
    See \cite[p. 35-36]{Tarapov} and \cite{Rodriguez07}.
\end{lemma}

With this we can define the configuration space of the snakes:
\begin{definition}[Configuration space]
    \[\Conf_{\mathcal{P}}(d-1) = \diffPw{0}{[0,L]}{\Sd{d-1}} \]
    is the \term{configuration space} of snakes of length $L$.
    It will be abbreviated as $\Conf$ in the following.
    Points $z \in \Conf$ are called \term{configurations}.
\end{definition}

\begin{remark}
    Observe that we can reconstruct a unique snake from a configuration $z$,
    since the tail is fixed:
    \[
        S_z(s) = \int_0^s z(\sigma) d\sigma
    \]
\end{remark}

With this we define the following for the rest of this text:
\begin{definition}
    \[f : \Conf \rightarrow \Rd, z \mapsto S_z(L)\]
\end{definition}

The image of $f$ is the set of all reachable points by all snakes of
length $L$ which is obviously a closed ball $L\Bd$ of size $L$:
\[
    L\Bd = \setDef{x \in \Rd}{\norm{x} \leq L}
\]

Now we can reformulate the problem:
\begin{problem}\label{prob:pathLift}
    Given an initial configuration $z \in \Conf$ and a curve
    $\gamma \in \diffRd{1}{[0,1]}$ with
    $\gamma(0) = f(z)$. Find a family of configurations $z_t$ \st $z_0 = z$ and
    $f(z_t) = \gamma(t)$.
\end{problem}

\begin{remark}\label{rem:path_lift}
    Because we are looking for a path
    $\widetilde{\gamma} : [0,1] \rightarrow \Conf, t \mapsto z_t$
    in the configuration space with $f \circ \widetilde{\gamma} = \gamma$ this
    is called a \term{path lifting problem}. See the following commutative diagram.

    \begin{center}
    \begin{tikzpicture}[node distance=3cm, auto]
      \node (I) {$[0,1]$};
      \node (Rd) [right of=I] {$\Rd$};
      \node (C) [above of=Rd] {$\Conf$};
      \draw[->] (C) to node {$f$} (Rd);
      \draw[->, dashed] (I) to node {$\widetilde{\gamma}$} (C);
      \draw[->] (I) to node [swap] {$\gamma$} (Rd);
    \end{tikzpicture}
    \end{center}
\end{remark}

