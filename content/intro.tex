%\chapter{Mathematical Foundations}\label{chap:math}

\section{Grundlegende Definitionen und Problembeschreibung}\label{sec:problem}

TODO: Einleitung erweitern

In diesem Abschnitt soll kurz in die zu behandelnde Problematik eingeführt
werden. Dazu werden die nötigen Begriffe definiert.

\begin{definition}
    Wir verwenden übliche \term{Netzwerke} bzw. gerichtete, endliche \term{Graphen}
    $\graph = (V, \A)$ bestehend aus Knoten und Kanten,
    wobei $n := |V|$ und $m := |\A|$.
    
    Einem solchen Netzwerk werden einige Funktionen zugeordnet:
    \begin{itemize}
        \item $\head : \A \rightarrow V$, $(u,v) \mapsto v$
        \item $\tail : \A \rightarrow V$, $(u,v) \mapsto u$
        \item $\outEdges : V \rightarrow 2^\A$,
            $v \mapsto \setDef{e \in \A}{tail(e) = v}$
        \item $\inEdges : V \rightarrow 2^\A$,
            $v \mapsto \setDef{e \in \A}{head(e) = v}$
    \end{itemize}
    
    Der Graph wird mit Funktionen angereichert, die mit als Eingabe
    für die Probleme dienen.
    \begin{itemize}
        \item $\tau : \A \rightarrow \R$, $e \mapsto \tau_e$ -- \term{Laufzeit}
        \item $u : \A \rightarrow \R$, $e \mapsto u_e$ -- \term{Kapazität}
        \item $c : \A \rightarrow \R$, $e \mapsto c_e$
            -- \term{Kosten} pro Flusseinheit
    \end{itemize}
\end{definition}

Zur Einführung werden wir zuerst den \term{statischen Fluss} definieren, dieser
hat im Gegensatz zum dynamischen keine Zeitkomponente.

\begin{definition}
    Sei $S \subseteq V$ die Menge der \term{Terminale} mit $S = S^+ \;\dot{\cup}\; S^-$
    und Funktionen
    \begin{itemize}
        \item $D : S^+ \rightarrow \R_+$ -- \term{Zulauf}
        \item $D : S^- \rightarrow \R_-$ -- \term{Bedarf}
    \end{itemize}
    die $\sum_{v \in S} D(v) = 0$ erfüllen. Wir nehmen dabei an, dass keine Kanten in
    Quellen hineinführen und keine aus Senken heraus: $\inEdges(S^+) = \emptyset$ und
    $\outEdges(S^-) = \emptyset$.
    
    Wenn $S^+ = \{s\}$ und $S^- = \{t\}$, dann schreiben wir kurz $d = D(s) = -D(t)$.
\end{definition}

\begin{remark}
    Um die Betrachtungen zu vereinfachen, wollen wir die Terminalknoten
    für jede Ware durch jeweils einen Knoten ersetzen. Dazu können
    Knoten $\setDef{s_i}{i \in K}$ und $\setDef{t_i}{i \in K}$ zu dem
    Netzwerk hinzugefügt werden. Diese werden mit den jeweiligen
    Terminalknoten verbunden: $(s_i, v)$ für alle $v \in S_i^+$ und
    $(v,t_i)$ für alle $v \in S_i^-$. Diesen Knoten wird die Kapazität $\infty$
    und Laufzeit $0$ zugeordnet. Diese Knoten werden zu den neuen Terminalen
    erklärt.

    Dann lassen sich die Bedarfsfunktionen pro Ware definieren:
    $D(i) = \sum_{v \in S_i^+} D(vi)$.

    Wir werden also lediglich Netzwerke betrachten, die pro Ware nur eine
    Quelle und Senke haben.

    % BjH: Diese Bemerkung bezieht sich auf mehrere Waren, was aber erst weiter unten eingeführt wird (insbesondere auch die Symbole mit den $i$-Indizes).
\end{remark}

\begin{definition}[Statischer Fluss mit einer Ware]
    Ein statischer Fluss ist eine Abbildung $x : \A \rightarrow \R_+$ mit
    \[ \sum_{e \in \outEdges(v)} x(e) - \sum_{e \in \inEdges(v)} x(e) = 0
        \quad, \text{alle } v \in V \setminus S. \]

    Ein solcher Fluss sollte im Allgemeinen folgende Eigenschaften erfüllen:
    \begin{itemize}
        \item Erfüllung von Bedarf und Zulauf:
            \[ \sum_{e \in \outEdges(v)} x(e) - \sum_{e \in \inEdges(v)} x(e) = D(v)
                \quad, \text{alle } v \in S. \]
        \item Zulässigkeit:
            \[ x(e) \leq u(e) \quad, \text{alle } e \in \A \]
    \end{itemize}

    Einem Fluss können außerdem die nötigen Kosten zugeordnet werden:
        \[ c(x) = \sum_{e \in \A} c(e) \cdot x(e). \]
\end{definition}

\begin{definition}[Mehrere Waren]
    Die oben definierten Flüsse können lediglich einen Bedarf erfüllen, will
    man unterschiedliche Waren zulassen, geht man zu sogenannten
    \term{Multicommodity}-Problemen über.

    Dazu wird eine Menge von Waren $K = \{1, \ldots, k\}$ sowie die jeweiligen
    Terminale $S_i = S_i^+ \disjUnion S_i^- \subseteq V$ festgelegt. Für diese
    Terminale muss dann jeweils der Bedarf angegeben werden:
    \begin{enumerate}
        \item $\func{D_i}{S_i^+}{\R_+}$ -- Zulauf
        \item $\func{D_i}{S_i^-}{\R_-}$ -- Bedarf
    \end{enumerate}

    Wir setzen $S := \bigcup\limits_{i \in K} S_i$.
\end{definition}

\begin{definition}[Statischer Fluss mit mehreren Waren]
    Die Waren induzieren eine Familie von statischen Flüssen
    $x = \{\func{x_i}{\A}{\R_+}\}_{i \in K}$. Dabei ist $x$ zulässig, wenn die
    Kapazität durch den Transport aller Waren nicht verletzt wird:
    \begin{align*}
        & \sum_{i \in K} x_i(e) \leq u(e) \quad \text{, alle } e \in \A.
    \end{align*}

    Die Kosten werden entsprechend auch über alle Waren aufsummiert,
    wobei die Kosten auch hier durch eine Familie von Funktionen
    $\{\func{c_i}{\A}{\R_+}\}_{i \in K}$ bestimmt werden.
    \begin{align*}
        & c(x) = \sum_{e \in A} \sum_{i \in K} c_i(e) \cdot x_i(e)
    \end{align*}
\end{definition}

Nachdem wir nun statische Flüsse ohne Zeitkomponente kennengelernt haben,
wollen wir Flüsse über Zeit (dynamische Flüsse) betrachten. Werden direkt
die allgemeinere Definition mit mehreren Waren geben.
% BjH: Im letzten Satz fehlt ein Wort.

\begin{definition}
    Ein \term{dynamischer Fluss} ist eine Familie von Funktionen
    $\{\func{f_i}{\A \times \ropen{0,T}}{\R_+}\}_{i \in K}$.
    $f_i(e,t)$ ist die \term{Flussrate}, die zur Zeit $t$ an der Kante $e$ anliegt.
    Dabei muss \\ $\lim_{t \to T} f_i(e,t) = 0$ gelten. Dann kann $f$ stetig
    auf $\ropen{T, +\infty}$ fortgesetzt werden.
    % BjH: Wo kommt das $\lim_{t \to T}$ her? Zumindest kann ich es im Paper nicht finden. Ein Limes setzt Konvergenz voraus; darf die Flussrate nicht pl\"otzlich auf Null gehen? (siehe auch lim unten.)

    Schreibe auch: $\func{f_i(e)}{\ropen{0,T}}{\R_+}$ mit $f_i(e)(t) = f_i(e,t)$.

    Um das Verhalten eines Flusses zu analysieren, definieren wir uns die Last an
    einem Knoten durch
    \[
        F_i(v, t) =
            \int_0^t \left(
                \sum_{e \in \outEdges(v)} f_i(e, \theta) -
                \sum_{e \in \inEdges(v)} f_i(e, \theta - \tau(e)) \right) \;d\theta .
    \]

    Der Fluss muss dann folgendes erfüllen:
    \begin{itemize}
        \item $F_i(v, t) \leq 0$ für alle $v \in S_i^-$ (Senken)
            und $t \in \ropen{0,T}$
        \item $F_i(v, t) = 0$ für alle $v \in V \setminus S$ und $t \in \ropen{0,T}$
            ohne Speicher bzw. $F_i(v, t) \leq 0$ wenn Speicher zugelassen ist.
            Dann muss aber auch $\lim_{t \to T} F_i(v, t) = 0$,
            damit kein Fluss zurückbleibt.
    \end{itemize}

    %Der Gesamtfluss auf einer Kante ist die Summe über alle Warenflüsse:
    %\[
    %    f(e, t) := \sum_{i \in K} f_i(e,t)
    %\]

    Ein dynamischer Fluss wird \term{zulässig} genannt, wenn der Gesamtfluss
    die Kapazität nicht übersteigt:
    \[
        \sum_{i \in K} f_i(e,t) \leq u(e) \quad \text{, alle } e \in \A
                                                \text{ und } t \in \ropen{0,T}.
    \]

    Auch einem dynamischen Fluss werden wieder Kosten zugeordnet:
    \[
        c(f) = \sum_{i \in K} \sum_{e \in \A} c_i(e) \int_0^T f_i(e,t) \: dt.
    \]
\end{definition}

\begin{remark}
    Der Fluss $f_i(e,t)$ kommt zur Zeit $t + \tau(e)$ bei $head(e)$ an. Daher folgt
    sofort, dass $f_i(e,t) = 0$, wenn $t \in \ropen{T - \tau(e), T}$.
\end{remark}

Nun wollen wir uns dem \term{Flusswert} zuwenden. Dieser ist insofern wichtig,
als dass er angibt, ob der Bedarf für eine Ware erfüllt ist.

\begin{definition}
    Seien $x$ ein statischer und $f$ ein dynamischer Fluss. Der \term{Flusswert}
    ist dann jeweils definiert durch

    \[
        |x| = \sum_{i \in K} \sum_{v \in S_i^+} \sum_{e \in \inEdges(v)} x_i(e)
    \]
    und
    \[
        |f| = \sum_{i \in K} \sum_{v \in S_i^+}
                \sum_{e \in \inEdges(v)} \int_0^T f_i(e,t) \: dt
            = \sum_{i \in K} \sum_{v \in S_i^+} F_i(v,T).
    \]
    Die zweite Gleichung gilt dabei, da ein Fluss eine Senke nicht wieder
    verlässt und Summe und Integral dann vertauscht werden dürfen.
\end{definition}

Das letzte Werkzeug, das wir benötigen werden, sind \term{Pfadzerlegungen}
von Flüssen.

\begin{definition}\label{def:path_flow}
    Seien $x$ ein statischer und $f$ ein dynamischer Fluss.
    Sei weiterhin $\pathSet$ eine endliche Menge von Pfaden von $S_i^+$ nach
    $S_i^-$. Wenn es eine Familie $\{\func{x_i}{\pathSet}{\R_+}\}_{i \in K}$
    bzw. $\{\func{f_i}{\pathSet \times \ropen{0,T}}{\R_+}\}_{i \in K}$ gibt,
    dann ist $\pathSet$ eine \term{Pfadzerlegung}, wenn gilt
    \[
        x_i(e) = \sum_{P \in \pathSet : e \in P} x(P) \quad \text{, für alle } e \in \A
    \]
    bzw.
    \[
        f_i(e, t) = \sum_{P \in \pathSet : e \in P} f(P, t - \tau(P \downarrow e)) .
    \]
    Dabei ist $P \downarrow e$ die Einschränkung von $P$ auf die Kanten \emph{vor} $e$
    und $\tau(P) = \sum_{e \in P} \tau(e)$ die natürliche Erweiterung von $\tau$ auf
    Pfade.

    Wenn Speicher zugelassen ist, müssen Verzögerungen bei dynamischen Flüssen
    mit einbezogen werden.
\end{definition}

