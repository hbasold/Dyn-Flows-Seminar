%\chapter{Mathematical Foundations}\label{chap:math}

\section{Grundlegende Definitionen und Problembeschreibung}\label{sec:problem}
%TODO: Einleitung erweitern
In diesem Abschnitt soll kurz in die zu behandelnde Problematik eingeführt
werden. Dazu werden die nötigen Begriffe definiert.

\begin{definition}
    Wir verwenden übliche \term{Netzwerke} bzw. gerichtete, endliche \term{Graphen}
    $\graph = (V, \A)$ aus Knoten und Kanten wobei $n := |V|$ und $m := |\A|$.
    
    Zu einem Graphen gibt es einige nützliche Funktionen:
    \begin{itemize}
        \item $\head : \A \rightarrow V$, $(u,v) \mapsto v$
        \item $\tail : \A \rightarrow V$, $(u,v) \mapsto u$
        \item $\outEdges : V \rightarrow 2^\A$, $v \mapsto \setDef{e \in \A}{tail(e) = v}$
        \item $\inEdges : V \rightarrow 2^\A$, $v \mapsto \setDef{e \in \A}{head(e) = v}$
    \end{itemize}
    
    Der Graph wird mit einigen Funktionen angereichert, die mit als Eingabe
    für die Probleme dienen.
    \begin{itemize}
        \item $\tau : \A \rightarrow \R$, $e \mapsto \tau_e$ -- \term{Laufzeit}
        \item $u : \A \rightarrow \R$, $e \mapsto u_e$ -- \term{Kapazität}
        \item $c : \A \rightarrow \R$, $e \mapsto c_e$ -- \term{Kosten} pro Flusseinheit
    \end{itemize}
\end{definition}

Zur Einführung werden wir zuerst den \term{statischen Fluss} definieren, dieser
hat im Gegensatz zum dynamischen keine Zeitkomponente.

\begin{definition}
    Sei $S \subseteq V$ die Menge der \term{Terminale} mit $S = S^+ \;\dot{\cup}\; S^-$
    und Funktionen
    \begin{itemize}
        \item $D : S^+ \rightarrow \R_+$ -- \term{Zulauf}
        \item $D : S^- \rightarrow \R_-$ -- \term{Bedarf}
    \end{itemize}
    die $\sum_{v \in S} D(v) = 0$ erfüllen.
    
    Wenn $S^+ = \{s\}$ und $S^- = \{t\}$, dann schreiben wir kurz $d = D(s) = -D(t)$.
\end{definition}

\begin{definition}[Statischer Fluss]
    Ein statischer Fluss ist eine Abbildung $x : \A \rightarrow \R_+$ mit
    \[ \sum_{e \in \outEdges(v)} x(e) - \sum_{e \in \inEdges(v)} x(e) = 0 \quad, \text{alle } v \in V \setminus S. \]
    
    Ein solcher Fluss sollte im Allgemeinen Folgende Eigenschaften erfüllen:
    \begin{itemize}
        \item Erfüllung von Bedarf und Zulauf:
            \[ \sum_{e \in \outEdges(v)} x(e) - \sum_{e \in \inEdges(v)} x(e) = D(v) \quad, \text{alle } v \in S. \]
        \item Zulässigkeit:
            \[ x(e) \leq u(e) \quad, \text{alle } e \in \A \]
    \end{itemize}
    
    Einem Fluss können außerdem die nötigen Kosten zugeordnet werden:
        \[ c(x) = \sum_{e \in \A} c(e) \cdot x(e). \]
\end{definition}
