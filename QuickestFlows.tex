\documentclass[alg,ngerman,seminar,pdf,article]{unitext}

\usepackage[utf8]{inputenc}

\usepackage{enumitem}
\makeatletter
\let\mdwNote\note                        % make \IEEEproof do same as \proof
\let\note\@undefined                        % undefine \proof
\makeatother

\usepackage{calligra}
\usepackage{amsmath}
\usepackage{amsthm}
\usepackage{stmaryrd}
\usepackage{float}
\usepackage{subfig}
\usepackage{framed}

\usepackage{algorithmic}
\usepackage{algorithm}
\floatname{algorithm}{\algorithmName}
\numberwithin{algorithm}{section}
\algsetup{linenosize=\normalsize, linenodelimiter=.}
\renewcommand{\algorithmicrequire}{\textbf{Eingabe:}}
\renewcommand{\algorithmicensure}{\textbf{Ausgabe:}}

%\usepackage{gnuplottex}

\usepackage{tikz}
\usepackage{environ}
\makeatletter
\newsavebox{\measure@tikzpicture}
\NewEnviron{scaletikzpicturetowidth}[1]{%
  \def\tikz@width{#1}%
  \def\tikzscale{1}\begin{lrbox}{\measure@tikzpicture}%
  \BODY
  \end{lrbox}%
  \pgfmathparse{#1/\wd\measure@tikzpicture}%
  \edef\tikzscale{\pgfmathresult}%
  \BODY
}
\makeatother

%\usepackage[final]{pdfpages}
\usepackage[draft]{pdfpages}

\newtheoremstyle{note} % name
                {10pt} %Space above               
                {10pt} %Space below
                {\itshape} %Body font
                {} %Indent amount
                {\bfseries} % Theorem head font
                {:} %Punctuation after theorem head
                {.5em} %Space after theorem head
                {} %Theorem head spec (can be left empty, meaning ‘normal’)

\newtheoremstyle{definition} % name
                {10pt} %Space above               
                {10pt} %Space below
                {} %Body font
                {} %Indent amount
                {\bfseries} % Theorem head font
                {} %Punctuation after theorem head
                {\newline} %Space after theorem head
                {} %Theorem head spec (can be left empty, meaning ‘normal’)

\newtheoremstyle{theorem} % name
                {10pt} %Space above               
                {10pt} %Space below
                {\itshape} %Body font
                {} %Indent amount
                {\bfseries} % Theorem head font
                {} %Punctuation after theorem head
                {.5em} %Normal Space after theorem head
                {} %Theorem head spec (can be left empty, meaning ‘normal’)

\newtheoremstyle{example} % name
                {10pt} %Space above               
                {10pt} %Space below
                {} %Body font
                {} %Indent amount
                {\bfseries} % Theorem head font
                {} %Punctuation after theorem head
                {\newline} %Space after theorem head
                {} %Theorem head spec (can be left empty, meaning ‘normal’)




\theoremstyle{definition}
%\newtheorem{definition}{Definition}[chapter]
\newtheorem{definition}{Definition}[section]
\newtheorem*{standaloneProof}{\proofName}

\theoremstyle{theorem}
\newtheorem{theorem}[definition]{\theoremName}
\newtheorem{lemma}[definition]{\lemmaName}
\newtheorem{corollary}[definition]{\corollaryName}
\newtheorem{problem}[definition]{\problemName}

\theoremstyle{example}
\newtheorem{example}[definition]{\exampleName}

\theoremstyle{remark}
\newtheorem{remark}[definition]{\remarkName}
\newtheorem{observation}[definition]{\observationName}

\renewcommand{\proofname}{\proofName}

%%% named references
\newcommand{\defRef}[1]{\definitionName \ \ref{def:#1}}
\newcommand{\refRemark}[1]{\remarkName \ \ref{#1}}
\newcommand{\theoRef}[1]{\theoremName \ \ref{theo:#1}}
\newcommand{\corRef}[1]{\corollaryName \ \ref{cor:#1}}
\newcommand{\lemRef}[1]{\lemmaName \ \ref{lem:#1}}
\newcommand{\probRef}[1]{\problemName \ \ref{prob:#1}}
\newcommand{\figRef}[1]{\figureName \ \ref{fig:#1}}
\newcommand{\algRef}[1]{\algorithmName \ \ref{alg:#1}}

%%% Highlighting and indexing of terms
\newcommand{\term}[1]{\emph{#1}\index{#1}}

% reduce space around cdot
\let\oldcdot=\cdot
\def\cdot{\negthinspace\oldcdot\negthinspace}

\def\notmid{\!\!\:\!\not|\;}

\newcommand{\R}{\mathbb{R}}
\newcommand{\Rd}{\R^d}
\newcommand{\Sd}[1]{\mathbb{S}_{#1}}
\newcommand{\Bd}{\mathbb{B}_{d}}
\newcommand{\I}{[0,1]}
\newcommand{\RdEx}{\widehat{\mathbb{R}}^d}

%% Möbius groups
\DeclareMathOperator{\GM}{GM}
\newcommand{\GMRd}{\GM(\Rd)}
\DeclareMathOperator{\M}{M}
\newcommand{\MRd}{\M(\Rd)}
\newcommand{\MSd}{\M(\Sd{d-1})}

%% Operators %%
\DeclareMathOperator{\grad}{grad}
\DeclareMathOperator{\im}{im}
\DeclareMathOperator{\Conf}{Conf}

%% sets of differentiable functions %%
\newcommand{\diffPw}[3]{C^{#1}_{\mathcal{P}}({#2}, {#3})}
\newcommand{\diffRdPw}[2]{\diffPw{#1}{#2}{\Rd}}
\newcommand{\diffRd}[2]{C^{#1}({#2}, \Rd)}
\newcommand{\diff}[3]{C^{#1}({#2}, {#3})}

%% Special functions/operators
\newcommand{\arclength}[1]{\ell({#1}, \mathcal{P})}
\newcommand{\norm}[1]{\left\Vert{#1}\right\Vert}
\newcommand{\inner}[2]{\left\langle{#1},{#2}\right\rangle}

%% Set definitions
\newcommand{\where}{\:|\:} % for set definitions
\newcommand{\setDef}[2]{\left\{{#1} \:|\: {#2}\right\}}


%%%%%%%% textual formatings %%%%%%%%%%
\newcommand{\resp}{\mbox{resp.}\ }
\newcommand{\st}{\mbox{s.t.}\ }


\graphicspath{{images/}}

\title{Quickest flows over time}
\author{Henning Basold}
%\date{October 17th, 2011}
%\dozent{Dr. Iris Reinbacher}
\betreuer{Björn Hendriks}

\renewcommand{\purposename}{Purpose}
\renewcommand{\abstractname}{Abstract}

%\keywords{Elliptic Curve Cryptography, Parallelization, Key Agreement}
%\makeglossary
%\makeindex
%\makenomenclature

\begin{document}

\titelblatt             %%% obligatorisch
%\zusammenfassung        %%% obligatorisch, in der Datei abstract.tex
%\cleardoublepage
\tableofcontents        %%% obligatorisch
%\listoftables           %%% optional
%\listoffigures          %%% optional
%\abkuerzung             %%% optional, in der Datei abbreviation.tex

\starttext
%\chapter{Mathematical Foundations}\label{chap:math}

\section{Problem description}\label{sec:problem}
The problem that will be tackled in this talk is called the ”Snake Charmer's Problem”.
It is about deforming a curve (the \emph{snake}) \st its \emph{snout} follows another
given curve (the snake charmer).

More precisely: a curve $S : [0,L] \rightarrow \Rd$ of length $L$ which is
piece-wise differentiable, has arc-length $L$ and fixed \term{tail} $S(0) = 0$ is called
a \term{snake}. $S(L)$ is called the \term{snout}.

Given a snake and a continuously differentiable curve $\gamma : [0,1] \rightarrow \Rd$.
Find a family $S_t, t \in [0,1]$ of snakes of length $L$ \st $S_0 = S$ and its
snouts follow $\gamma$: $S_t(L) = \gamma(t)$.

In the following the necessary definitions will be given to understand the above
description fully and to be able to reformulate the problem into a solvable
problem.

Everything will be formulated in a way that the topic is understandable without
having deep knowledge about Topology, manifolds and so on. Only a geometric imagination
of spheres, planes and curves in $\Rd$ (mostly $\R^3$) is needed. The purpose of this
report and talk is to demonstrate the power of topological abstraction in solving
problems coming from computer science.


\section{Zeiterweiterte Netzwerke}

Unser Ziel ist ein polynomielles Approximationsschema für Quickest-Flow-Probleme.
Die Idee dabei ist, ein Netzwerk $\graph$ in ein \term{zeit-erweitertes Netzwerk}
umzuwandeln. Dazu werden für jeden Zeitschritt Kopien der Knoten von $\graph$
angelegt. Dabei gibt die Zeitkomponente den Zeitpunkt eines Flusses an. Daher
werden in dem neuen Graph zwei Knoten verbunden, genau dann wenn die Knoten
im ursprünglichen Graphen durch eine Kante $e \in \A$ verbunden gewesen sind und
sie $\tau(e)$ Zeitschritte entfernt sind. Zusätzlich müssen wir noch Kanten
zwischen den einzelnen Zeitschritte der Quellen und Senken einfügen, damit
der Fluss auch am Ende der Zeit des Flusses verfügbar ist, denn wir werden
die Kopien der Senken zum Zeitpunkt $T-1$ als neue Senken erklären.

Wir nehmen dabei an, dass die Zeiten ganzzahlig sind. Dies ist durch Skalierung
der Zeit auf die gewünschte Genauigkeit für Anwendungen problemlos möglich.

\begin{definition}[zeiterweitertes Netzwerk]
    Sei $(\graph, \tau)$ ein Netzwerk mit $\tau(e) \in \N$ für alle $e \in \A$
    und $T \in \N$. Setze $\timeDom = \{0, 1, \ldots, T-1\}$.
    Wir definieren damit das \term{zeiterweiterte Netzwerk} $\tExp{T} = (V^T, \A^T)$
    mit
    \begin{itemize}
        \item $V^T = V \times \timeDom$ (schreibe $v_t = (v, t)$ und
            $V_t = \{v_t \in V^T\}$)
        \item $\A^T = \setDef{e_t = (v_t, u_{t + \tau(e)})}
                        {e = (v, u) \in \A \text{ und } t, t + \tau(e) \in \timeDom}
                    \cup H$ \\
            mit „\term{holdover arcs}“ bzw. \term{Speicherkanten} \\
            $H = \setDef{(v_t, v_{t+1})}
                        {v \in S_i \text{ und } t, t + 1 \in \timeDom}$
        \item $\func{u^T}{\A^T}{\REx = \R \cup \{\infty\}}$ mit
                $u^T(e_t) = \begin{cases}
                    u(e)    &, e_t \not\in H \\
                    \infty  &, e_t \in H
                \end{cases}$
        \item $\func{c^T}{\A^T}{\R}$ mit
                $c^T(e_t) = \begin{cases}
                    c(e)    &, e_t \not\in H \\
                    0       &, e_t \in H
                \end{cases}$
        \item $S_i^{+,T} = \setDef{v_0 \in V^T}{v \in S_i^+}$, \\
                $S_i^{-,T} = \setDef{v_{T-1} \in V^T}{v \in S_i^-}$ und damit \\
                $S_i^T = S_i^{+,T} \disjUnion S_i^{-,T}$
    \end{itemize}
    Sollte Speicher an Knoten $V \setminus S$ zugelassen sein, dann müssen analog
    zu $H$ weitere Speicherkanten eingefügt werden.
\end{definition}

\begin{remark}
    Da in Netzwerken $\inEdges(s) = \emptyset$ für alle $s \in S_i^+$ und
    $\outEdges(t) = \emptyset$ für alle $t \in S_i^-$ (d.h. Quellen und Senken
    können keine inneren Knoten auf Flüssen sein), kann in der obigen Definition
    auch $\REx$ durch $\R$ ersetzt werden, wenn man die Kapazität der
    Speicherkanten $(v_t, v_{t+1}) \in H$ größer oder gleich $|D_i(v)|$ für
    $v \in S_i$ wählt. Wenn der Fluss diesen Bedarf übersteigen darf,
    muss die Kapazität entsprechend größer gewählt werden.
\end{remark}

\begin{example}
    Wir wollen uns ein einfaches Beispiel für ein zeiterweitertes Netzwerk 
    ansehen. Dazu sehen wir uns (Bild ..) an. In diesem deklarieren wir
    nur $s$ als Quelle und $t$ als Senke: $S^+ = \{s\}, S^- = \{t\}$.
    Wir betrachten $T = 6$ und damit $\timeDom = \{0, \ldots, 5\}$. An den
    Kanten sind die Zeiten $\tau$ notiert. Wir nehmen außerdem an, dass
    an den Knoten $u$ und $v$ Speicher zugelassen ist. Dann ergibt sich
    des Netzwerk $\tExp{T}$ in (Bild ...).
\end{example}

Da wir dynamische Flüsse in mit Hilfe von Algorithmen für statische Flüsse
konstruieren wollen, müssen wir zwischen Flüssen $f$ in $\graph$ und
Flüssen $x$ in $\tExp{T}$ wechseln können:

\begin{lemma}\label{lem:static_dyn_conv}
    Ein statischer Fluss $x$ in $\tExp{T}$ entspricht einem dynamischen Fluss
    $\func{f(e)}{\ropen{0,T}}{\R_+}$ in $\graph$ mit gleichem Flusswert und Kosten
    und umgekehrt. Dabei gilt:
    \begin{enumerate}
        \item $f_i(e,t) := x_i(e_{\lfloor t \rfloor})$ und
        \item $x_i(e_t) := \int_t^{t+1} f_i(e, \theta) d\theta$
    \end{enumerate}
        
    
    \begin{proof}
        TODO: S. 8,9.
    \end{proof}
\end{lemma}

Dieser Sachverhalt macht zwar sehr einfach, hat allerdings das Problem, dass das
Netzwerk $\tExp{T}$ sehr groß wird. Denn $|V^T| = |V| \cdot |\timeDom| = n \cdot T$.
Da aber $T$ in der Größe $log T$ kodiert wird, hängt $V^T$ nicht mehr polynomiell
von der Eingabe ab.

Die Idee ist nun, das Netzwerk $\tExp{T}$ wieder zu verkleinern, so dass es wieder
eine polynomielle Größe erreicht, indem unnötige Zeitschritte entfernt werden.
Das führt zu den \term{skalierten Netzwerken}:

\begin{definition}
    Sei $(\graph, \tau)$ Netzwerk und $0 < \Delta$, so dass $T/\Delta \in \N$
    und $\tau(e)/\Delta \in \N$ für alle $e \in \A$. Wir setzen
    $\tau'(e) := \tau(e)/\Delta$ für alle $e \in \A$
    und dann $\redNetw{T}{\Delta} := \tExp{T/\Delta}$ für $(\graph, \tau')$.
    
    Ein Knoten $v_t$ entspricht dann einem Zeitverlauf im Intervall
    $\ropen{t \Delta, (t+1) \Delta}$. Daher müssen die Kosten und Kapazitäten
    auf $c(e) \cdot \Delta$ und $u(e) \cdot \Delta$ korrigiert werden.
\end{definition}

\begin{lemma}
    Sei $0 < \Delta$ und $\graph$ wie oben. Dann entspricht ein statischer Fluss $x$
    in $\redNetw{T}{\Delta}$ einem Fluss $\func{f(e)}{\ropen{0, T}}{\R_+}$ in
    $\graph$ mit gleichen Kosten und umgekehrt.
    
    \begin{proof}
        Wende \lemRef{static_dyn_conv} auf $x$, dies ergibt einen Fluss
        $\func{\hat{f}(e)}{\ropen{0, T/\Delta}}{\R_+}$. Mit Skalierung der
        Zeit um $\Delta$ ergibt sich $f$. Der Rest ist klar.
    \end{proof}
\end{lemma}







\section{Gleichförmig reduzierte zeit-expandierte Netzwerke}\label{sec:unif_cond}

TODO: Definiere Fluss auf Pfaden!

\subsection{Quickest Transshipment mit einer Ware (singlecommodity)}

Zu QTP:
\begin{theorem}\label{theo:qtp_flow_ex}
    Sei $T \geq T^*$. Setze $\Delta := \frac{\eps^2 T}{n}$ und
    $T' := \lceil (1+\eps)^3 T / \Delta \rceil \Delta$.
    Dann gilt:
    \begin{enumerate}
        \item in $\redNetw{T'}{\Delta}$ existiert ein statischer Fluss $x$ mit
            Bedarf $|x| = (1 + \eps)D$ und Kosten $c(x) \leq (1+\eps)C$.
        \item aus $x$ kann ein dynamischer Fluss $f$ in $\graph$ berechnet werden,
            der bei $(1+\eps)T'$ endet und für den $|f| = D$ und $c(f) \leq C$
            gilt.
    \end{enumerate}
\end{theorem}

Bevor wir diesen Satz beweisen, folgern wir daraus den Algorithmus:

TODO:
\begin{theorem}
    QTP-FPTAS-Core
\end{theorem}

TODO: Einbettung in Suchframework

\begin{theorem}\label{theo:slow_flow}
    Sei $\func{f(e)}{\ropen{0, T + \delta}}{\R_+}$ ein dynamischer Fluss in $(\graph, \tau)$
    mit $|f| = D$ und $c(f) \leq C$ und einer endlichen Zerlegung in Flüsse $f(P)$
    auf Pfaden $P \in \pathSet$. Sei $\eps > 0$ und $(\graph, \tau')$ das
    gleiche Netzwerk mit geänderten Zeiten, für die
    $\left|\tau(e) - \tau'(e)\right| \leq \frac{\eps^2T}{n}$ gilt. Definiere
    einen Fluss in $(\graph, \tau')$
    \[
    \tilde{f}(P)(t) := \frac{1}{1+\eps} \frac{1}{\eps T}
                            \int_{t - \eps T}^{t} f(P)(\theta) \; d\theta \text{ .}
    \]
    Dann gilt für $\tilde{f}$:
    \begin{enumerate}
        \item $\tilde{f}$ endet bei $\delta + T + \eps T + \eps^2 T$
        \item $\tilde{f}$ verläuft nur auf Pfaden aus $\pathSet$
        \item $|\tilde{f}| = \frac{D}{1 + \eps}$
        \item $c(\tilde{f}) \leq \frac{C}{1 + \eps}$
    \end{enumerate}
    
    \begin{proof}
        TODO: in Tex und Anpassung an $T + \delta$.
    \end{proof}
\end{theorem}

\begin{lemma}\label{lem:relaxed_flow}
    Sei $\func{f^*(e)}{\ropen{0,T^*}}{\R_+}$ ein dynamischer Fluss in $(\graph, \tau)$
    mit Bedarf $|f^*| = D$ und Kosten $c(f^*) \leq C$. Für alle $\delta \geq 1$ und
    und $T \geq T^*$ existiert ein dynamischer Fluss
    $\func{f(e)}{\ropen{0, \delta T}}{\R_+}$ mit $|f| = \delta D$ und
    $c(f) \leq \delta C$. Dabei ist $f$ kreisfrei/kommt ohne Speicher aus, wenn $f^*$
    dies ist/tut.
    
    \begin{proof}
        TODO: in Tex
    \end{proof}
\end{lemma}

Zu QTP:
\begin{theorem}\label{theo:qtp_opt_flow}
    TODO: Existenz optimaler Fluss mit Pfadzerlegung und ohne Speicher
     (Corollary 4.5).
\end{theorem}

\begin{standaloneProof}[\theoRef{qtp_flow_ex}a]
    Nach \theoRef{qtp_opt_flow} existiert ein optimaler Fluss $f^*$, der
    keinen Speicher benötigt und nur auf Pfaden verläuft.
    Setze $\delta = (1 + \eps)^2$ und wende \lemRef{relaxed_flow} mit
    $T \geq T^*$ an und erhalte einen Fluss $f$ der bei $(1 + \eps)^2 T$ endet.
    $f$ besitzt damit eine endliche Pfadzerlegung, hat Bedarf
    $|f| = (1 + \eps)^2 D$ und Kosten $c(f) \leq (1 + \eps)^2 C$.
    
    Setze nun $\tau'(e) := \lceil \tau(e) / \Delta \rceil \Delta$ für
    alle $e \in \A$. Dann ist offensichtlich
    $|\tau(e) - \tau'(e)| \leq \Delta = \frac{\eps^2 T}{n}$.
    Da $\delta T = T + 2 \eps T + \eps^2 T$ ist, liefert
    \theoRef{slow_flow} uns einen Fluss $\tilde{f}$ in $(\graph, \tau')$
    der bei
    \begin{align*}
        & T + 2 \eps T + \eps^2 T + \eps T + \eps^2 T \\
        & = T + 3 \eps T + 2 \eps^2 T \\
        & \leq T + 3 \eps T + 3 \eps^2 T + \eps^3 T \\
        & = (1 + \eps)^3 T \leq T'
    \end{align*}
    endet. $\tilde{f}$ hat Bedarf
    $|\tilde{f}| = \frac{|f|}{1+\eps} = (1 + \eps) D$ und ebenso
    Kosten $c(\tilde{f}) \leq (1 + \eps) C$.
    
    Da $\tau'$ und $T'$ nach Konstruktion ganzzahlig durch
    $\Delta$ teilbar sind, lässt sich $\tilde{f}$ als statischer Fluss $x$ in
    $\redNetw{T'}{\Delta}$ interpretieren, der die gewünschten
    Eigenschaften hat.
    
    \begin{flushright}\qed \end{flushright}
\end{standaloneProof}

\begin{lemma}\label{lem:path_decomp}
    Eine Pfadzerlegung $\pathSet$ von einem statischen Fluss $x$ in
    $\redNetw{T'}{\Delta}$ kann in eine Pfadzerlegung $\pathSet'$ des
    zugehörigen dynamischen Flusses $f'$ in $(\graph, \tau')$ 
    transformiert werden. Dabei wird $\ropen{0, T'}$ in $T'/\Delta$
    disjunkte Teilintervalle zerlegt, auf den $f'(P)$ auf allen $P \in \pathSet'$
    konstant ist.
    
    \begin{proof}
        TODO: in Tex
    \end{proof}
\end{lemma}

\begin{standaloneProof}[\theoRef{qtp_flow_ex}b]
    Der Fluss $x$ in $\redNetw{T'}{\Delta}$ induziert einen Fluss $f'$ in
    $(\graph, \tau')$:
    \begin{align*}
        & \func{f'(e)}{\ropen{0,T'}}{\R_+} \\
        & f'(e)(t) = x(e)(\lfloor t \rfloor) \\
        & \quad \text{ (d.h. } f'(e)(\theta) = x(e)(t)
            \text{ für alle } \theta \in \ropen{t, t+1} \text{ ) }
    \end{align*}
    Offensichtlich ist $f'$ ein zulässiger Fluss in $(\graph, \tau')$ und
    hat die gleichen Eigenschaften wie $x$ (Bedarf $(1+\eps)D$, Kosten $(1+\eps)C$).
    
    Mit \lemRef{path_decomp} bekommen wir eine Pfadzerlegung $\pathSet'$.
    Der erste Teil von \theoRef{slow_flow} liefert uns wieder einen Fluss
    $f$ auf $\ropen{0, (1 + \eps)T'}$ mit Bedarf $D$ und Kosten
    $c(f) \leq C$.
    
    TODO: evtl. \theoRef{slow_flow} aufteilen.
    
    \begin{flushright}\qed \end{flushright}
\end{standaloneProof}

TODO: Laufzeitanalyse

\begin{remark}
    Alles was wir für den Beweis von \theoRef{qtp_flow_ex} benutzt haben, ist
    unabhängig von einem speziellen Flussproblem gültig
    (bis auf \theoRef{qtp_opt_flow}). Dies liefert uns also ein „Framework“
    zur Konstruktion von Algorithmen und entsprechenden Beweisen.
\end{remark}

\subsection*{Quickest Transshipment mit mehren Waren (multicommodity)}
Dies wollen wir direkt ausnutzen, um ein das obige Verfahren auf mehrere
Waren zu verallgemeinern. Die Approximation hiervon ohne Speicher ist allerdings
NP-hart (TODO: Quelle [19] dazu betrachten), da eine optimale Lösung ohne Speicher
Zyklen enthalten kann. Mit Speicher kann allerdings einfach gewartet werden,
anstatt einen Zyklus zu benutzen.

Allerdings kann aus einem Pfad, auf dem Speicher verwendet wird, nicht mehr wie
in dem Beweis von \theoRef{slow_flow} der Fluss auf Kanten rekonstruiert werden.
Das kommt daher, dass dort erwartet wird, das sich die Laufzeit eines Flusses
direkt aus $\tau$ ergibt. Mit Speicher ist dies aber nicht der Fall
(s. dazu Definition von $\tau(P \downarrow e)$).

TODO: definiere $P^\delta$ und $\tau(P^\delta \downarrow e)$, lasse dabei $\delta$
eine Funktion $V \to \R_+$ sein.

TODO: verallgemeinere \theoRef{slow_flow}, um Speicher mit dieser Definition zuzulassen
und setze in den Beweisen im letzten Abschnitt $\delta = 0$

Dann folgt dies alles als Korollar.


\section{Nicht gleichförmig reduzierte zeiterweiterte Netzwerke}\label{sec:nonunif_cond}
Die gleichmäßige Unterteilung der Zeit durch $\timeDom = \{0,1, \ldots, T-1\}$ kann bei
anderen Problemen zu schlechten Ergebnissen führen. Wir wollen dazu kurz das Problem
des „Earliest Arrival Flows“ und dazu ein kritisches Beispiel betrachten.

TODO: Problem

TODO: Beispiel + Bild

Man kann dieses Problem umgehen, indem man den Zeitverlauf anders unterteilt. In dem
Beispiel würde die Unterteilung in Intervalle der Größe $\frac{1}{2}$
genügen ($\timeDom = \{0, \frac{1}{2}, 1\}$). Wir wollen uns für diesen Fall nur
die entsprechende Definition des zeitertweiterten Netzwerks ansehen.

\begin{definition}[Zeiterweitertes Netzwerk mit beliebigen Zeitintervallen]
    Sei $(\graph, \tau)$ und $L = (\theta_q)_{q \in R}$, wobei $R = \{0, \ldots, r\}$,
    sodass
    \[
        0 = \theta_0 < \theta_1 < \ldots < \theta_r < T.
    \]
    Setze $\theta_{r+1} = T$.
    Dann ist das \term{L-zeiterweiterte Netzwerk} $\tExp{L} = (V^L, \A^L)$
    gegeben durch
    \begin{itemize}
        \item $V^L = V \times R$ (schreibe $v_q = (v, q)$ und
            $V_q = \{v_q \in V^L\}$)
        \item $\A^L = \setDef{e_q = (v_q, u_{m(e, q)})}
                        {e = (v, u) \in \A, q \in R, \theta_q + \tau(e) \leq \theta_r}
                    \cup H$, \\
            \text{ wobei } $H = \setDef{(v_q, v_{q+1})}
                                    {v \in S_i \text{ und } q, q+1 \in R}$
    \end{itemize}

    Dabei ist $\func{m}{\A \times R}{R}$ definiert durch
    \[
        m(e, q) := \min \setDef{q' \in R}{\theta_q + \tau(e) \leq \theta_{q'}}.
    \]
    $m$ bestimmt also das Intervall, zwischen welchen die Kante $e$ transportiert.
    Der Rest ist analog wie bei den ursprünglichen zeiterweiterten Netzwerken
    definiert.
\end{definition}

\begin{remark}
    Offensichtlich ergibt sich mit einer Unterteilung $L = (0, 1, \ldots, T-1)$
    wieder $\tExp{T} = \tExp{L}$. Also sind L-zeiterweiterte Netzwerke
    eine Verallgemeinerung.
\end{remark}

TODO: wie werden diese benutzt?

%%%% Einträge im Glossar werden durch den Befehl
%%%    \glossar{begriff}{erklärung}
%%% vorgenommen.

\glossar{Gruppe}{Eine \emph{Gruppe} ist eine
Menge mit einer zweistelligen assoziativen Verknüpfung mit Einselement
und inversen Elementen.}

\glossar{Abelsche Gruppe}{Eine Gruppe
heißt \emph{abelsch}, wenn die Verknüpfung kommutativ ist.}

\glossar{Ä}{ein Eintrag, der mit einem Umlaut beginnt.}


\nocite{*}
\bibliography{lit}
%\printglossary
%\printindex
\anhang

\end{document}
